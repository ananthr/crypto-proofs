\documentclass[A4, 11pt]{article}
\usepackage{mathptmx}

\input{/Users/ananthr/tex-macros/macros.tex}

\begin{document}

\title{Proofs in Cryptography\footnote{{\bf Rough Draft} of a handout for
CS255: Introduction to Cryptography by Dan Boneh.}}
\author{Ananth Raghunathan\thanks{{\tt ananthr@stanford.edu}}}
\date{}
\maketitle

\begin{abstract}
We give a brief overview of proofs in cryptography at a beginners level. We
briefly cover a general way to look at proofs in cryptography and briefly
compare the requirements to traditional reductions in computer science. We
then look at two security paradigms, indistinguishability and simulation
based security. We also describe the security models for Secret Key and
Public Key systems with appropriate motivations. Finally, we cover some
advanced topics and conclude with a few exercises and hints.
\end{abstract}

\section{Introduction}

\subsection{General Outline}

\subsection{Relation to reductions in Complexity Theory}

\subsection{Two security paradigms*}

\subsubsection{IND*}

\subsubsection{SIM*}

\pagebreak
\section{Secret Key Cryptography}

\subsection{Information Theoretic Arguments}

\subsection{Semantic Security}

\subsection{PRGs}

\subsection{PRFs}

\subsection{Block Ciphers or Pseudorandom Functions}

\subsection{Message Authentication Codes}

\subsection{Collision-resistant Hash functions}

\section{Public Key Cryptography}

\subsection{Semantic Security against CPA}

\subsection{CCA}

\subsection{OWFs}

\subsection{Signature schemes}

\section{Advanced Topics*}

\subsection{Hybrid Arguments*}

\subsection{Randomized Self Reductions*}

\section{Exercises}

\begin{enumerate}
\item Prove that truncating the output of a PRF is still secure
\item Prove that any subset of the bits of a PRG are still indistinguishable
\item Prove that a $\ell$-expanding PRG is a small domain PRF
\item Introduce hardness of SVP in perp lattices and prove that
$f_A(x)=A\cdot x \imod{q}$ is a collision resistant hash function
\item Prove that a PRF implies a One Way Function
\end{enumerate}
\end{document}
