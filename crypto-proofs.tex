\documentclass[A4, 11pt]{article}
\usepackage{mathptmx}

\input{/Users/ananthr/tex-macros/macros.tex}

\newcommand{\todo}[1]{\small{\sc #1}\normalsize \\}

\begin{document}

\title{Proofs in Cryptography\footnote{{\bf Rough Draft} of a handout for
CS255: Introduction to Cryptography by Dan Boneh.}}
\author{Ananth Raghunathan\thanks{{\tt ananthr@stanford.edu}}}
\date{}
\maketitle

\begin{abstract}
We give a brief overview of proofs in cryptography at a beginners level. We
briefly cover a general way to look at proofs in cryptography and briefly
compare the requirements to traditional reductions in computer science. We
then look at two security paradigms, indistinguishability and simulation
based security. We also describe the security models for Secret Key and
Public Key systems with appropriate motivations. Finally, we cover some
advanced topics and conclude with a few exercises and hints.
\end{abstract}

\section{Introduction}
\todo{Talk about the need for working with reductions and different security
definitions. The subtleties that arise in definitions, the time it took to
come up with good definitions of security.}

Before we go about describing a general outline to prove the security of
constructions, let us have an informal discussion. Proving the
impossibility of a particular computational task is extremely difficult. It
is very close to the P vs.~NP question (add reference) that is the central
question in computer science today. It becomes doubly difficult when do not
know what class of computational tasks we are allowed to utilize (unlike
the case of P vs.~NP, where a Turing machine has been rigorously and
beautifully formalized). Therein lies the central difficulty in proving the
security of cryptographic constructions. 

One way to address the latter question is to explicitly model what an
adversary who intends to break our cryptosystem is allowed to do. Notice
this requires us to define at least two things: what do we mean by an
``adversary'' and what it means to ``break'' the cryptosystem. A lot of
emperical and theoretical work goes into the current models of
``adversary'' and ``break'' and currently the definitions we use have been
used widely in practice and give us good results emperically. 

However, as it happens often in this field, the adversary in the real world
is not constrained to behave as we dictate him\footnote{or her. I will
restrict myself to male pronouns for simplicity.}. There have been several
real world attacks (side-channel attacks, cold-boot attacks are recent
innovative attack (add references)) that completely side-step the
adversarial model and as expected completely breaks the security of the
scheme. However, this only serves as motivation to model stronger
adversaries and thereby construct more robust cryptosystems that are
provably robust (with respect to the new models).

\par
In conclusion, remember two important things: To analyse a cryptosystem you
must define an adversary model and a security game. 

\begin{enumerate}
\item {\bf Adversary model:} This defines formally the power of the
adversary. It includes specifics whether the adversary is
deterministic/randomized, uniform/non-uniform, interactive/non-interactive
and how he interacts with the security game. 

\item {\bf Security game:} This defines formally the power the adversary
has over the cryptosystem. Whether he has access to a single ciphertext,
multiple ciphertexts, multiple keys, etc. It also defines when an adversary
is said to break the system. 
\end{enumerate}
Often adversaries are modeled similarly, but depending on whether you want
weak or strong security guarantees, you modify the security game
accordingly.

Two broad paradigms for security games are indistinguishablity games and
simulation games. Subsection \ref{subsec:indsim} will talk about it in more
detail. 

\subsection{General Outline}
\subsection{Relation to reductions in Complexity Theory}

\subsection{Two security paradigms*} \label{subsec:indsim}

\subsubsection{IND*}

\subsubsection{SIM*}

\section{Secret Key Cryptography}

\subsection{Information Theoretic Arguments}

\subsection{Semantic Security}

\subsection{PRGs}

\subsection{PRFs}

\subsection{Block Ciphers or Pseudorandom Functions}

\subsection{Message Authentication Codes}

\subsection{Collision-resistant Hash functions}

\section{Public Key Cryptography}

\subsection{Semantic Security against CPA}

\subsection{CCA}

\subsection{OWFs}

\subsection{Signature schemes}

\section{Advanced Topics*}

\subsection{Hybrid Arguments*}

\subsection{Randomized Self Reductions*}

\section{Exercises}

\begin{enumerate}
\item Prove that truncating the output of a PRF is still secure
\item Prove that any subset of the bits of a PRG are still indistinguishable
\item Prove that a $\ell$-expanding PRG is a small domain PRF
\item Introduce hardness of SVP in perp lattices and prove that
$f_A(x)=A\cdot x \imod{q}$ is a collision resistant hash function
\item Prove that a PRF implies a One Way Function
\end{enumerate}
\end{document}
